\section{DM10: Science Platform Portal}\label{sect:dm10}

A detailed run through specific requirements for the portal is given in \appref{sect:reqs}.
Many of the front end requirements are currently met and others may be questioned, however invoking this option introduces some risk that science may be lost.
We should be clear that effectively removing one of our development groups is disruptive to the DM project,
 especially after a
review which approved the new budget and schedule which we have been following well.

Concretely we would reduce the effort expenditure on the Portal and Firefly from IPAC to about 0.2 FTE over a period of four to five months.
We would continue the IPAC support for DM Architecture and the Science Platform Scientist.

If we follow DM10 to the letter, we would eliminate the Portal and this would mean a reduction in support of the broad science community,
especially novice and casual users.
Hence we would rather keep the portal in its current incarnation, pause development, and then restart closer
to the initial need date for a portal (DR1).
At this point, we would take the opportunity to evaluate the then-available technology for the Portal.
% The portal work started too early on LSST by many estimations.

The LSST EPO subsystem also needs a portal and has already decided that
Firefly is not a suitable technology. A consideration of possible alternatives
is already underway.
Within the AURA family, STScI and NOAO have portals.
We have concerns about their capabilities and usability
\newtext{and believe that the current version of the LSST Portal is more advanced and suited to science goals of LSST},
but perhaps those could be addressed with appropriate directives from AURA and synergy with LSST.

The existing system is adequate for the analysis of commissioning data.
We would therefore target the availability of DR1 for the launch of an updated
portal.
The other area of potential concern is the front end for the alert subscription in the mini broker.
This should be \newtext{a} thin veneer on the mini broker API and is also not needed until start of operations.

Given the above we would reserve \$1M for the later development of an updated portal for DR1
We would not go to zero immediately to allow IPAC to adjust accordingly.
Plus we would keep some level of effort --- say, 0.2 FTE --- for bug fixes on the existing portal.
An LCR will provided detailed costings, but initial estimates are that this will save at least \$2M.
We believe the final impact on science is minimal.
