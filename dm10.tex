\section{DM10  Science Platform Portal }\label{sect:dm10}

A detailed run through specific requirements for the portal is given in \appref{sect:reqs}.

Concretely we would drop the portal and firefly support from  IPAC  to about 0.2 FTE over a period of four to five months.
We would continue the IPAC support for DM Architecture and the Science Platform scientist.

If we follow DM10 to the letter
we would eliminate the portal and this would mean a reduction in support of the broad science community, especially the less capable and casual users.
Hence we would rather keep the portal we have, pause development, and restart closer to the initial need date for a portal (DR1).
After a two year pause we could reevaluate available portal technology.
% The portal work started too early on LSST by many estimations.

Within LSST EPO also need a portal and already decided Firefly was not a suitable technology, hence the search for alternatives is already underway within LSST.
Within the AURA family STScI and NOAO have portals, admittedly they are not very good ones, but perhaps that could  change with appropriate directives from AURA and provide potential synergy.

When to start up again is an interesting question - we should make some commissioning data available for example. The existing portal should be ok for that so the suggestion would be target DR1 for an updated portal.  The other area of potential concern is the front end for the alert subscription in the mini broker - this should be  thin veneer on the mini broker API and is also not needed until start of operations.

Given the above  we would keep \$1M for the later development of an updated portal for DR1. We would not go to zero immediately to allow IPAC to adjust accordingly. Plus we would keep some level of effort like 0.2 for bug fixes on the existing portal. An LCR would give the exact numbers but this should save \$2M or more. We believe the final impact on science is minimal.

